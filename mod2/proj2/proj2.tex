\documentclass[12pt]{article}

\usepackage[a4paper]{geometry} %page size
\usepackage{parskip} %no paragraph indentation
\usepackage{fancyhdr} %fancy stuff in page header
\pagestyle{fancy} 

\usepackage[utf8]{inputenc} %encoding
\usepackage[danish]{babel} %danish letters

\usepackage{graphicx} %import pictures
\graphicspath{ {images/} }
\usepackage{listings} %make lists

\usepackage{amsmath, amssymb, amsfonts, amsthm, mathtools} %doing math
\usepackage{algorithmicx, algpseudocode} %doing pseudocode

\title{
  Title\
  \large Subtitle
}
\author{Asger Andersen}
\date{\today}

\fancyhead{}
\lhead{This is the title}
\rhead{Asger Andersen}

%End of preamble
%*******************************************************************************

\begin{document}

\section{Opgave 1: Kapitalforretning}

Jeg antager i hele denne opgave, at renten $r$ er skarpt større end 0.

\subsection{Delopgave a)}

Lad $x_t$ være vores kapitals størrelse i år $t$ og lad $r_t$ være renten i år. Hvis vi ingen penge bruger i år $t$, så vil vores kapital året efter (altså år $t+1$) være på 
\begin{align}
x_{t+1} = (1+r_t)x_t
\end{align}
kroner. Hvis vi udtrækker $u_t$ af kapitalen i løbet af år $t$, så vil vores kapital året efter - ifølge min logik - være på
\begin{align}
x_{t+1} = (1+r_t)(x_t - u_t)
\end{align}
kroner, da vi ikke burde få rente af den kapital, vi udtrækker. Dette svarer dog ikke helt til opgaveteksten, hvor vi tilsyneladende får rente på vores kapitals størrelse ved starten af år $t$ (før udtrækket $u_t$), da formlen her er
\begin{align}
x_{t+1} = (1+r_t)x_t - u_t
\end{align}
Jeg ved ikke meget om renter, så måske er min forståelse forkert og opgavetekstens differensligning rent faktisk en mere korrekt beskrivelse af, hvordan renter typisk fungerer.

\subsection{Delopgave b)} 

Ved konstant forretning og linært voksende udtrækning fås differensligningen
\begin{align}
x_{t+1} = (1+r)x_t - u
\end{align}
Ifølge boks 10 på side 195 i kursusbogen, er den fuldstændige løsning til denne differensligning givet ved
\begin{align}
x_t = k(1+r)^t + \frac{-u}{1 - (1+r)} = k(1+r)^t + \frac{u}{r}
\end{align}
hvor 
\begin{align}
k = x_0 - \frac{-u}{1 - (1+r)} = x_0 - \frac{u}{r}
\end{align}
Det er klart ud fra formlen for den fuldstændige løsning af differensligningen, at $x_t$ bliver negativ på et tidspunkt, hvis og kun hvis $k<0$. Dette er tilfældet, hvis og kun hvis
\begin{align}
x_0 < \frac{u}{r}
\end{align} 
Hvis vi for eksempel har, at renten er på 1 procent og vi trækker 1000 kroner ud årligt, så skal vores startkapital være på 100000 kroner eller derover, hvis vi aldrig skal gå i overtræk.

Her er et plot for værdierne af $x_t$ for $t=0,...,10$, når $r=0.015$, $u=12000$ og $x_0=200000$:
\begin{center}
\includegraphics[scale=0.5]{q1p1.png}
\end{center}

\subsection{Delopgave c)}

Fra boks 8 side 195 i kursusbogen fås, at den fuldstændige løsning til differensligningen med konstant forretning og linært voksende udtræk er givet ved:

\begin{align}
x_t = (1+r)^{t-1}\left(x_0 + \sum_{\tau=0}^{t-1}\frac{-(u + \alpha\tau)}{(1+r)^\tau} \right)
\end{align}

Her er et plot for værdierne af $x_t$ for $t=0,...,10$, når $r=0.015$, $u=1000$, $\alpha=4000$ og $x_0=200000$:
\begin{center}
\includegraphics[scale=0.5]{q1p2.png}
\end{center}

Jeg har brugt uniroot i r til numerisk at bestemme, at hvis $r=0.015$, $u=1000$ og $x_0=200000$, så er $x_10>0$, hvis og kun hvis $\alpha<4652.81$.

\subsection{Delopgave d)}

Vi får nu differensligningen
\begin{align}
x_{t+1} = (1 + r_0r^t)x_t - (u + \alpha t)
\end{align}

Her er et plot for værdierne af $x_t$ for $t=1,...,10$, når $r_0=0.02$, $r=0.9$, $u=1000$, $\alpha=4000$ og $x_0=200000$:
\begin{center}
\includegraphics[scale=0.5]{q1p3.png}
\end{center}

\subsection{Delopgave e)}

Vi har nu denne differensligning
\begin{align}
x_{t+1} = \exp(r_0r^t)x_t
\end{align}
hvor $r_0=0.02$ og $r=0.9$. Ifølge boks 4 på side 194 i kursusbogen, har denne differensligning den fuldstændige løsning
\begin{align}
x_0E_t
\end{align}
hvor $x_0\in \mathbb{R}$ og 
\begin{align}
E_t = \prod_{\tau=0}^{t-1}\left( \exp(r_0r^\tau) \right) =  \exp\left(\sum_{i=0}^{t-1}r_0r^\tau\right) = \exp\left(r_0\frac{1-r^\tau}{1-r}\right)
\end{align}
Altså har differensligningen den fuldstændige løsning
\begin{align}
x_0 \exp\left(r_0\frac{1-r^\tau}{1-r}\right)
\end{align}
Den specifikke løsning for $x_0=200000$ er klar ud fra den fuldstændige løsning.

\section{Opgave b: Nationaløkonomisk model}

Lad
\begin{align}
A = \begin{bmatrix}
a  && a\\
(a-1)c && ac
\end{bmatrix}
\end{align}

Fra miniprojekt 1 ved vi, at det karakteristiske polynomien for $A$ er givet ved
\begin{align}
\rho(\lambda) = \lambda^2 + (-a(c+1))\lambda + ac
\end{align}

For $a=0.8$ og $c=3$ har dette polynomien rødderne
\begin{align}
\lambda_1=\frac{6}{5},\qquad \lambda_2 = 2
\end{align}
som altså er egenværdier for $A$. For at finde en egenvektor $q$ hørende til $\lambda_1$ skal vi løse ligningen
\begin{align}
(A-\lambda_1 E)q = 0
\end{align}
Det kan vi gøre ved rækkeoperationer
\begin{align}
\begin{bmatrix}
-\frac{2}{5} && \frac{4}{5} && 0\\
\\
\frac{-3}{5} && \frac{6}{5} && 0
\end{bmatrix}
\\
\begin{bmatrix}
-2 && 4 && 0\\
\\
-3 && 6 && 0
\end{bmatrix}
\\
\begin{bmatrix}
-2 && 4 && 0\\
\\
0 && 0 && 0
\end{bmatrix}
\\
\begin{bmatrix}
1 && -2 && 0\\
\\
0 && 0 && 0
\end{bmatrix}
\end{align}
Altså er egenvektorrummet for $\lambda_1$ givet ved
\begin{align}
\{(q_1, q_2)\in \mathbb{R}^2\ |\ q_1 = 2q_2 \}
\end{align}
Alle vektorer i dette rum er egenvektorer for $A$ hørende til egenværdien $\lambda_1$. Et eksempel på en sådan egenvektor er altså $v = (1, 2)$.

På tilsvarende vis kan vi løse ligningen
\begin{align}
(A-\lambda_2 E)q = 0
\end{align}
for at finde ud af, at egenvektorrummet for $\lambda_2$ er givet ved
\begin{align}
\{(q_1, q_2)\in \mathbb{R}^2\ |\ 3q_1 = 2q_2 \}
\end{align}
Alle vektorer i dette rum er egenvektorer for $A$ hørende til egenværdien $\lambda_2$. Et eksempel på en sådan egenvektor er altså $w = (3, 2)$.

Fra miniprojekt 1 ved jeg, at den givne nationaløkonomiske model har netop én ligevægt givet ved
\begin{align}
\begin{pmatrix}
C^*\\
I^*
\end{pmatrix}
=\begin{pmatrix}
\frac{b}{1-a}\\
0
\end{pmatrix}
\end{align}
For $a=0.8$ og $b=6$ har modellen altså denne ligevægt
\begin{align}
\begin{pmatrix}
C^*\\
I^*
\end{pmatrix}
=\begin{pmatrix}
30\\
0
\end{pmatrix}
\end{align}
Altså har den nationaløkonomiske model den fuldstændige løsning
\begin{align}
\begin{pmatrix}
C_t\\I_t
\end{pmatrix} = c_1\left(\frac{6}{5} \right)^t \begin{pmatrix} 1 \\ 2\end{pmatrix} +
c_2 2^t \begin{pmatrix} 3 \\ 2\end{pmatrix} + \begin{pmatrix}
30 \\ 0
\end{pmatrix}
\end{align}
hvor $c_1$ og $c_2$ er vilkårlige reelle tal. For at bestemme den partikulære løsning med $(C_0, I_0)=(40,9)$ skal vi løse ligningen
\begin{align}
\begin{pmatrix}
40\\9
\end{pmatrix} = c_1\left(\frac{6}{5} \right)^0 \begin{pmatrix} 1 \\ 2\end{pmatrix} +
c_2 2^0 \begin{pmatrix} 3 \\ 2\end{pmatrix} + \begin{pmatrix}
30 \\ 0
\end{pmatrix}
\end{align}
hvilket er ækvivalent med ligningen
\begin{align}
\begin{pmatrix}
10\\9
\end{pmatrix} = \begin{bmatrix}
1 && 3\\
2 && 2
\end{bmatrix} \begin{pmatrix}
c_1 \\ c_2
\end{pmatrix}
\end{align}
Vi kan for eksempel opstille totalmatricen for denne ligning og bruge rækkeoperationer til at finde ud af, at 
\begin{align}
\begin{pmatrix}
c_1\\c_2 
\end{pmatrix} = \begin{pmatrix}
\frac{7}{4}\\ \frac{11}{4}
\end{pmatrix}
\end{align}
Indsætter vi disse konstanter i den fuldstændige løsning, så får vi den søgte partikulære løsning.

\section{Opgave 3: Epidemimodel}

\subsection{Delopgave a}

Der er tale om et autonomt system af differensligninger, da tiden t kun påvirker værdien af $(S_{t+1}, I_{t+1})$ gennem værdien af $(S_t, I_t)$. Der er ikke tale om et linært system af differensligninger, da $S_{t+1}$ ikke er en linær funktion af $(S_t, I_t)$, da størrelserne $S_t^2$ og $S_tI_t$ indgår (dette kan ses ved at gange parantesen ud).

\subsection{Delopgave b, c \& d}

Lad $0 < N, b$ og $0 < a,b < 1$. 

$(S^*,I^*)\in \mathbb{R}^2$ er en ligevægt for modellen, hvis og kun hvis
\begin{align}
S^* = (1-a)S^* + bS^*\left(1 - \frac{S^* + I^*}{N} \right)\\
I^* = (1-c)I^* + aS^*
\end{align}
Det er let at se, at uanset værdien af $a, b, c$ og $N$, så vil $(S^*,I^*)=(0,0)$ altid være en løsning af disse ligninger. Ved hjælp af kedsommelige regnerier kan vi desuden komme frem til, at hvis $(S^*,I^*)\neq(0,0)$, så er $(S^*,I^*)$ en løsning til ovenstående ligningssystem, hvis og kun hvis
\begin{align}
S^* = c\Gamma(N,a,b,c), \quad I^* = a\Gamma(N,a,b,c)
\end{align}
hvor vi har defineret
\begin{align}
\Gamma(N,a,b,c) =  \frac{N(b-a)}{b(a+c)}
\end{align}
Vi ser, at modellen har en ligevægt $(S^*, I^*)$, hvor $0<S^*, I^*$, hvis og kun hvis $b>a$.

Dette passer også med, at sætter vi $N=100000, a=0.3, b=1.8$ og $c=0.1$ og fremskriver værdierne af $(S_t, I_t)$ ud fra startvektoren $(S_0, I_0)=(1000,0)$, så får vi dette plot
\begin{center}
\includegraphics[scale=0.5]{q3p1.png}
\end{center}
hvor den blå graf er antallet af syge og den røde er antallet af immune. Her er $a>b$ og modellen ser ud til at have en ligevægt med $S^*$ omkring 21000 og $I^*$ omkring 62000. Laver vi samme plot med $b=0.2$, så får vi
\begin{center}
\includegraphics[scale=0.5]{q3p2.png}
\end{center}
Her er $b<a$ og modellen ser ud til at have ligevægten $(S^*, I^*)=(0,0)$. 

Lad os beregne alle ligevægte for $N=100000, a=0.3, b=1.8$ og $c=0.1$. Som sagt har vi altid ligevægten $(S^*, I^*)=(0,0)$. Bruger vi det generelle resultat fra linje (34-35), får vi desuden ligevægten $(S^*, I^*)=(20833.\overline{33},62500)$. Dette passer med, hvad vi så i plottet ovenfor.

Jeg vil nu finde ud af, om disse ligevægte er stabile. Jeg udregner funktionalmatricen for vores epidemimodel til
\begin{align}
\begin{bmatrix}
1 - a + b - b\frac{2S + I}{N} && - \frac{bS}{N} \\
\\
1-c && a 
\end{bmatrix}
\end{align}
Indsætter vi $N=100000, a=0.3, b=1.8, c=0.1, S=0$ og $I=0$ i funktionalmatricen, får vi matricen
\begin{align}
\begin{bmatrix}
2.5 && 0 \\
\\
0.9 && 0.3 
\end{bmatrix}
\end{align}
Den har egenværdierne $2.5$ og $0.3$, hvilket vil sige, at ligevægten $(0,0)$ ikke er lokalt stabilt, da $1\leq |2.5|$.

Indsætter vi $N=100000, a=0.3, b=1.8, c=0.1, S=20833.\overline{33}$ og $I=62500$ i funktionalmatricen, får vi matricen
\begin{align}
\begin{bmatrix}
0.625 && 0.375 \\
\\
0.9 && 0.3 
\end{bmatrix}
\end{align}
Den har egenværdierne $0.4625 + 0.5\overline{57}i$ og $0.4625 - 0.5\overline{57}i$, som begge har modulus $\sqrt{0.4625^2 + 0.5\overline{57}^2}=0.725<1$. Altså er ligevægten $(20833.\overline{33}, 62500)$ lokalt stabil.

\subsection{Delopgave e}

Lad $0 < N, b$ og $0 < a,b < 1$. 

$(S^*,I^*)\in \mathbb{R}^2$ er en ligevægt for den modificerede model, hvis og kun hvis
\begin{align}
S^* = (1-a)S^* + b(S^*)^2\left(1 - \frac{S^* + I^*}{N} \right)\\
I^* = (1-c)I^* + aS^*
\end{align}
Igen er det let at se, at uanset værdien af $a, b, c$ og $N$, så vil $(S^*,I^*)=(0,0)$ altid være en løsning af disse ligninger. 

For at bestemme andre løsninger til systemet, kan vi starte med at omskrive til det ækvivalente ligningssystem
\begin{align}
0 = -a + bS^*\left(1 - \frac{S^* + I^*}{N} \right)\\
I^* = \frac{c}{a}S^*
\end{align}
Indsætter vi udtrykket for $I^*$ fra den nederste ligning i den øverste, får vi
\begin{align}
0 = -a + bS^*\left(1 - \frac{S^* + \frac{c}{a}S^*}{N} \right)
\end{align}
Vi ser at dette er en andengradsligning og omskriver den ved hjælp af en masse regnerier til den konventionelle form
\begin{align}
0 = \left(\frac{bc}{aN} + \frac{b}{N} \right)(S^*)^2 + bS^* + a
\end{align}
Hermed får vi diskriminanten
\begin{align}
d = b\left(\frac{bN - 4(a+c)}{N}\right)
\end{align}
som er skarpt større end 0, hvis og kun hvis
\begin{align}
bN > 4(a+c)
\end{align}
Hvis dette er opfyldt, så får vi altså - udover nulløsningen - to løsninger givet ved
\begin{align}
S^* = \frac{b \pm \sqrt{d}}{2\left(\frac{bc}{aN} + \frac{b}{N} \right)}\\
I^* = \frac{c}{a}S^*
\end{align}
Disse er forskellige fra nulløsningen, hvis og kun hvis
\begin{align}
\sqrt{d} \notin \{-b,b\}
\end{align}
Hvis linje (42) og (45) er sande, så har modellen altså to ligevægte, der er forskellige fra $(0,0)$. 

PS: Jeg tror, at jeg har regnet forkert i denne delopgave, da betingelser, jeg kommer frem til, ikke ser ud til at være ækvivalente med den betingelse på relationen mellem $a,b,c$ og $N$, som opgaveteksten siger, at vi skal gøre os. Jeg vil dog mene, at min fremgangsmåde burde føre til det korrekte resultat, hvis ellers jeg kunne finde ud af at regne ordentligt :)

\end{document}