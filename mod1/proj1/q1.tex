\documentclass[]{article}
\usepackage{lmodern}
\usepackage{amssymb,amsmath}
\usepackage{ifxetex,ifluatex}
\usepackage{fixltx2e} % provides \textsubscript
\ifnum 0\ifxetex 1\fi\ifluatex 1\fi=0 % if pdftex
  \usepackage[T1]{fontenc}
  \usepackage[utf8]{inputenc}
\else % if luatex or xelatex
  \ifxetex
    \usepackage{mathspec}
  \else
    \usepackage{fontspec}
  \fi
  \defaultfontfeatures{Ligatures=TeX,Scale=MatchLowercase}
\fi
% use upquote if available, for straight quotes in verbatim environments
\IfFileExists{upquote.sty}{\usepackage{upquote}}{}
% use microtype if available
\IfFileExists{microtype.sty}{%
\usepackage{microtype}
\UseMicrotypeSet[protrusion]{basicmath} % disable protrusion for tt fonts
}{}
\usepackage[margin=1in]{geometry}
\usepackage{hyperref}
\hypersetup{unicode=true,
            pdfborder={0 0 0},
            breaklinks=true}
\urlstyle{same}  % don't use monospace font for urls
\usepackage{color}
\usepackage{fancyvrb}
\newcommand{\VerbBar}{|}
\newcommand{\VERB}{\Verb[commandchars=\\\{\}]}
\DefineVerbatimEnvironment{Highlighting}{Verbatim}{commandchars=\\\{\}}
% Add ',fontsize=\small' for more characters per line
\usepackage{framed}
\definecolor{shadecolor}{RGB}{248,248,248}
\newenvironment{Shaded}{\begin{snugshade}}{\end{snugshade}}
\newcommand{\KeywordTok}[1]{\textcolor[rgb]{0.13,0.29,0.53}{\textbf{#1}}}
\newcommand{\DataTypeTok}[1]{\textcolor[rgb]{0.13,0.29,0.53}{#1}}
\newcommand{\DecValTok}[1]{\textcolor[rgb]{0.00,0.00,0.81}{#1}}
\newcommand{\BaseNTok}[1]{\textcolor[rgb]{0.00,0.00,0.81}{#1}}
\newcommand{\FloatTok}[1]{\textcolor[rgb]{0.00,0.00,0.81}{#1}}
\newcommand{\ConstantTok}[1]{\textcolor[rgb]{0.00,0.00,0.00}{#1}}
\newcommand{\CharTok}[1]{\textcolor[rgb]{0.31,0.60,0.02}{#1}}
\newcommand{\SpecialCharTok}[1]{\textcolor[rgb]{0.00,0.00,0.00}{#1}}
\newcommand{\StringTok}[1]{\textcolor[rgb]{0.31,0.60,0.02}{#1}}
\newcommand{\VerbatimStringTok}[1]{\textcolor[rgb]{0.31,0.60,0.02}{#1}}
\newcommand{\SpecialStringTok}[1]{\textcolor[rgb]{0.31,0.60,0.02}{#1}}
\newcommand{\ImportTok}[1]{#1}
\newcommand{\CommentTok}[1]{\textcolor[rgb]{0.56,0.35,0.01}{\textit{#1}}}
\newcommand{\DocumentationTok}[1]{\textcolor[rgb]{0.56,0.35,0.01}{\textbf{\textit{#1}}}}
\newcommand{\AnnotationTok}[1]{\textcolor[rgb]{0.56,0.35,0.01}{\textbf{\textit{#1}}}}
\newcommand{\CommentVarTok}[1]{\textcolor[rgb]{0.56,0.35,0.01}{\textbf{\textit{#1}}}}
\newcommand{\OtherTok}[1]{\textcolor[rgb]{0.56,0.35,0.01}{#1}}
\newcommand{\FunctionTok}[1]{\textcolor[rgb]{0.00,0.00,0.00}{#1}}
\newcommand{\VariableTok}[1]{\textcolor[rgb]{0.00,0.00,0.00}{#1}}
\newcommand{\ControlFlowTok}[1]{\textcolor[rgb]{0.13,0.29,0.53}{\textbf{#1}}}
\newcommand{\OperatorTok}[1]{\textcolor[rgb]{0.81,0.36,0.00}{\textbf{#1}}}
\newcommand{\BuiltInTok}[1]{#1}
\newcommand{\ExtensionTok}[1]{#1}
\newcommand{\PreprocessorTok}[1]{\textcolor[rgb]{0.56,0.35,0.01}{\textit{#1}}}
\newcommand{\AttributeTok}[1]{\textcolor[rgb]{0.77,0.63,0.00}{#1}}
\newcommand{\RegionMarkerTok}[1]{#1}
\newcommand{\InformationTok}[1]{\textcolor[rgb]{0.56,0.35,0.01}{\textbf{\textit{#1}}}}
\newcommand{\WarningTok}[1]{\textcolor[rgb]{0.56,0.35,0.01}{\textbf{\textit{#1}}}}
\newcommand{\AlertTok}[1]{\textcolor[rgb]{0.94,0.16,0.16}{#1}}
\newcommand{\ErrorTok}[1]{\textcolor[rgb]{0.64,0.00,0.00}{\textbf{#1}}}
\newcommand{\NormalTok}[1]{#1}
\usepackage{graphicx,grffile}
\makeatletter
\def\maxwidth{\ifdim\Gin@nat@width>\linewidth\linewidth\else\Gin@nat@width\fi}
\def\maxheight{\ifdim\Gin@nat@height>\textheight\textheight\else\Gin@nat@height\fi}
\makeatother
% Scale images if necessary, so that they will not overflow the page
% margins by default, and it is still possible to overwrite the defaults
% using explicit options in \includegraphics[width, height, ...]{}
\setkeys{Gin}{width=\maxwidth,height=\maxheight,keepaspectratio}
\IfFileExists{parskip.sty}{%
\usepackage{parskip}
}{% else
\setlength{\parindent}{0pt}
\setlength{\parskip}{6pt plus 2pt minus 1pt}
}
\setlength{\emergencystretch}{3em}  % prevent overfull lines
\providecommand{\tightlist}{%
  \setlength{\itemsep}{0pt}\setlength{\parskip}{0pt}}
\setcounter{secnumdepth}{0}
% Redefines (sub)paragraphs to behave more like sections
\ifx\paragraph\undefined\else
\let\oldparagraph\paragraph
\renewcommand{\paragraph}[1]{\oldparagraph{#1}\mbox{}}
\fi
\ifx\subparagraph\undefined\else
\let\oldsubparagraph\subparagraph
\renewcommand{\subparagraph}[1]{\oldsubparagraph{#1}\mbox{}}
\fi

%%% Use protect on footnotes to avoid problems with footnotes in titles
\let\rmarkdownfootnote\footnote%
\def\footnote{\protect\rmarkdownfootnote}

%%% Change title format to be more compact
\usepackage{titling}

% Create subtitle command for use in maketitle
\newcommand{\subtitle}[1]{
  \posttitle{
    \begin{center}\large#1\end{center}
    }
}

\setlength{\droptitle}{-2em}
  \title{}
  \pretitle{\vspace{\droptitle}}
  \posttitle{}
  \author{}
  \preauthor{}\postauthor{}
  \date{}
  \predate{}\postdate{}


\begin{document}

\subsection{R code for question 1}\label{r-code-for-question-1}

\begin{Shaded}
\begin{Highlighting}[]
\KeywordTok{setwd}\NormalTok{(}\StringTok{'~/Desktop/Matematik og modeller/mod1/proj1'}\NormalTok{)}

\KeywordTok{library}\NormalTok{(tidyverse)}
\end{Highlighting}
\end{Shaded}

\begin{verbatim}
## -- Attaching packages --------- tidyverse 1.2.1 --
\end{verbatim}

\begin{verbatim}
## √ ggplot2 2.2.1     √ purrr   0.2.4
## √ tibble  1.4.2     √ dplyr   0.7.4
## √ tidyr   0.8.0     √ stringr 1.3.0
## √ readr   1.1.1     √ forcats 0.3.0
\end{verbatim}

\begin{verbatim}
## -- Conflicts ------------ tidyverse_conflicts() --
## x dplyr::filter() masks stats::filter()
## x dplyr::lag()    masks stats::lag()
\end{verbatim}

\begin{Shaded}
\begin{Highlighting}[]
\KeywordTok{library}\NormalTok{(xtable)}

\NormalTok{sim_mod <-}\StringTok{ }\ControlFlowTok{function}\NormalTok{(move, n, x1) \{}
  
\NormalTok{  xs <-}\StringTok{ }\KeywordTok{matrix}\NormalTok{(}\DataTypeTok{nrow =}\NormalTok{ n, }\DataTypeTok{ncol =} \KeywordTok{length}\NormalTok{(x1))}
\NormalTok{  xs[}\DecValTok{1}\NormalTok{,] <-}\StringTok{ }\NormalTok{x1}
  \ControlFlowTok{for}\NormalTok{ (t }\ControlFlowTok{in} \DecValTok{1}\OperatorTok{:}\NormalTok{(n}\OperatorTok{-}\DecValTok{1}\NormalTok{)) \{}
\NormalTok{    xs[t}\OperatorTok{+}\DecValTok{1}\NormalTok{,] <-}\StringTok{ }\KeywordTok{move}\NormalTok{(xs[t,])}
\NormalTok{  \}}
  
\NormalTok{  sim <-}\StringTok{ }\KeywordTok{data_frame}\NormalTok{(}\DataTypeTok{t =} \DecValTok{1}\OperatorTok{:}\NormalTok{n, }\DataTypeTok{C=}\NormalTok{xs[,}\DecValTok{1}\NormalTok{], }\DataTypeTok{I=}\NormalTok{xs[,}\DecValTok{2}\NormalTok{])}
\NormalTok{  plt <-}\StringTok{ }\NormalTok{sim }\OperatorTok\StringTok{ }
\StringTok{    }\KeywordTok{gather}\NormalTok{(key, value, C, I) }\OperatorTok\StringTok{ }
\StringTok{    }\KeywordTok{ggplot}\NormalTok{(}\KeywordTok{aes}\NormalTok{(t, value, }\DataTypeTok{color=}\NormalTok{key)) }\OperatorTok{+}\StringTok{ }
\StringTok{    }\KeywordTok{geom_line}\NormalTok{() }\OperatorTok{+}\StringTok{ }
\StringTok{    }\KeywordTok{theme}\NormalTok{(}\DataTypeTok{legend.title =} \KeywordTok{element_blank}\NormalTok{())}
  
  \KeywordTok{list}\NormalTok{(}\StringTok{'sim'}\NormalTok{ =}\StringTok{ }\NormalTok{sim, }\StringTok{'plt'}\NormalTok{ =}\StringTok{ }\NormalTok{plt)}
  
\NormalTok{\}}

\NormalTok{make_matrix <-}\StringTok{ }\ControlFlowTok{function}\NormalTok{(a,b,c) \{}
  \KeywordTok{matrix}\NormalTok{(}\KeywordTok{c}\NormalTok{(a, (a}\OperatorTok{-}\DecValTok{1}\NormalTok{)}\OperatorTok{*}\NormalTok{c, a, a}\OperatorTok{*}\NormalTok{c), }\DataTypeTok{ncol =} \DecValTok{2}\NormalTok{)}
\NormalTok{\}}

\NormalTok{a <-}\StringTok{ }\FloatTok{0.48}
\NormalTok{b <-}\StringTok{ }\DecValTok{1}
\NormalTok{c <-}\StringTok{ }\FloatTok{1.5}
\NormalTok{A <-}\StringTok{ }\KeywordTok{make_matrix}\NormalTok{(a,b,c)}
\NormalTok{A}
\end{Highlighting}
\end{Shaded}

\begin{verbatim}
##       [,1] [,2]
## [1,]  0.48 0.48
## [2,] -0.78 0.72
\end{verbatim}

\begin{Shaded}
\begin{Highlighting}[]
\NormalTok{move <-}\StringTok{ }\ControlFlowTok{function}\NormalTok{(x) \{}
\NormalTok{  A}\OperatorTok\NormalTok{x }\OperatorTok{+}\StringTok{ }\KeywordTok{c}\NormalTok{(b, b}\OperatorTok{*}\NormalTok{c)}
\NormalTok{\}}

\NormalTok{n <-}\StringTok{ }\DecValTok{50}
\NormalTok{x1 <-}\StringTok{ }\KeywordTok{c}\NormalTok{(}\FloatTok{1.8}\NormalTok{,}\FloatTok{0.4}\NormalTok{)}
\NormalTok{sim <-}\StringTok{ }\KeywordTok{sim_mod}\NormalTok{(move, n, x1)}

\NormalTok{sim[[}\StringTok{'plt'}\NormalTok{]]}
\end{Highlighting}
\end{Shaded}

\includegraphics{q1_files/figure-latex/unnamed-chunk-1-1.pdf}

\begin{Shaded}
\begin{Highlighting}[]
\NormalTok{sim[[}\StringTok{'sim'}\NormalTok{]][(n}\OperatorTok{-}\DecValTok{5}\NormalTok{)}\OperatorTok{:}\NormalTok{n,]}
\end{Highlighting}
\end{Shaded}

\begin{verbatim}
## # A tibble: 6 x 3
##       t     C          I
##   <int> <dbl>      <dbl>
## 1    45  1.92 -0.000291 
## 2    46  1.92 -0.000279 
## 3    47  1.92 -0.000126 
## 4    48  1.92  0.0000502
## 5    49  1.92  0.000151 
## 6    50  1.92  0.000145
\end{verbatim}

\begin{Shaded}
\begin{Highlighting}[]
\KeywordTok{print}\NormalTok{(}\KeywordTok{xtable}\NormalTok{(sim[[}\StringTok{'sim'}\NormalTok{]][(n}\OperatorTok{-}\DecValTok{5}\NormalTok{)}\OperatorTok{:}\NormalTok{n,], }\DataTypeTok{digits=}\DecValTok{6}\NormalTok{), }\DataTypeTok{file=}\StringTok{'q1tbl.tex'}\NormalTok{)}
\end{Highlighting}
\end{Shaded}


\end{document}
