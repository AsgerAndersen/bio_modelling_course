\documentclass[12pt]{article}

\usepackage[a4paper]{geometry} %page size
\usepackage{parskip} %no paragraph indentation
\usepackage{fancyhdr} %fancy stuff in page header
\pagestyle{fancy} 

\usepackage[utf8]{inputenc} %encoding
\usepackage[danish]{babel} %danish letters

\usepackage{graphicx} %import pictures
\graphicspath{ {images/} }
\usepackage{listings} %make lists

\usepackage{amsmath, amssymb, amsfonts, amsthm, mathtools} %doing math
\usepackage{algorithmicx, algpseudocode} %doing pseudocode

\title{
  Title\
  \large Subtitle
}
\author{Asger Andersen}
\date{\today}

\fancyhead{}
\lhead{This is the title}
\rhead{Asger Andersen}

%End of preamble
%*******************************************************************************

\begin{document}

\section{Opgave 1: Kaniner og ræve}

Vi har modellen
\begin{align}
\begin{pmatrix}
K'(t) \\ 
R'(t)
\end{pmatrix} = f(K, R)
\end{align}
hvor 
\begin{align}
f(K,R) = \begin{pmatrix}
r_K K (1 - \alpha K - bR) \\ 
r_R R (cK - 1)
\end{pmatrix}
\end{align}
hvor $\alpha>0$, $b=0.05$, $c=0.005$ og $r_K = r_R = 2$.

\subsection{Delopgave a}

Hvis vi antager $R(t)=0$ for alle $t$, så forsimples modellen til
\begin{align}
f(K,R) = \begin{pmatrix}
r_K K (1 - \alpha K) \\ 
0
\end{pmatrix} = \begin{pmatrix}
r_K K \left(1 - \frac{K}{B}\right) \\ 
0
\end{pmatrix} 
\end{align}
hvor $B=1/\alpha$. Altså får vi en model, hvor der ingen ræve er, og hvor antallet af kaniner vokser logistisk med vækstraten $r_k=2$ og befolkningskapaciteten $B=1/\alpha$.

\subsection{Delopgave b}

\begin{enumerate}
\item Rævebestanden falder lige efter starttidspunktet.
\item Da er rævebestanden stigende og kaninbestanden faldende.
\item Der synes at være en ligevægt med omtrent 200 kaniner og omtrent 15 ræve.
\item Ligevægten ser lokalt stabil ud, da det den snurrende bevægelse ind mod ligevægten tyder på, at der i hvert er et åbent område omkring ligevægten, hvorindenfor enhver kombination af de to befolkningsstørrelser med tiden trækkes mod ligevægten som mod centrummet af en malstrøm.
\end{enumerate}

\subsection{Delopgave c}

\begin{enumerate}
\item Jeg udregner \begin{align}
f(K(0), R(0)) = \begin{pmatrix}
2\cdot 100 \left(1 - \frac{100}{1000} - \frac{5\cdot 10}{100} \right) \\ \\
2 \cdot 10 \left( \frac{5\cdot 100}{1000} - 1\right)
\end{pmatrix} = 
\begin{pmatrix}
80 \\
-10
\end{pmatrix}
\end{align}

Altså er $R'(0)=-10$, hvilket vil sige, at rævebestanden er faldende til starttidspunktet 0.
\item  Jeg udregner \begin{align}
f(300, 20) = \begin{pmatrix}
2\cdot 300 \left(1 - \frac{300}{1000} - \frac{5\cdot 20}{100} \right) \\ \\
2 \cdot 20 \left( \frac{5\cdot 300}{1000} - 1\right)
\end{pmatrix} = 
\begin{pmatrix}
-180 \\
20
\end{pmatrix}
\end{align}
Altså er kaninbestanden stigende og rævebestanden faldende, når der er 300 kaniner og 20 ræve.
\end{enumerate}

\subsection{Delopgave d \& e}

Jeg ønsker at bestemme, om modellen har ligevægte $(K^*, R^*)$, hvor $K^*, R^*>0$. Derfor vil forsøge at løse systemet
\begin{align}
f(K^*, R^*) = (0,0), \qquad K^*, R^*>0
\end{align}
hvilket vi kan skrive ud som
\begin{align}
\begin{pmatrix}
r_K K^* (1 - \alpha K^* - bR^*) \\ 
r_R R^* (cK^* - 1)
\end{pmatrix} =
\begin{pmatrix}
0 \\ 0
\end{pmatrix}, \qquad K^*, R^*>0
\end{align}
Da $r_K, r_R, K^*, R^*>0$, får vi det ækvivalente system
\begin{align}
\begin{pmatrix}
1 - \alpha K^* - bR^* \\ 
cK^* - 1
\end{pmatrix} =
\begin{pmatrix}
0 \\ 0
\end{pmatrix}, \qquad K^*, R^*>0
\end{align}
Den nederste ligning giver os, at
\begin{align}
K^* = \frac{1}{c}, \qquad K^* >0
\end{align}
Indsætter vi det i den æverste ligning, får vi
\begin{align}
1 - \alpha \frac{1}{c} - bR^*, \qquad R^*>0
\end{align}
hvilket er ækvivalent med
\begin{align}
R^* = \frac{1}{b}\left( 1 - \frac{\alpha}{c} \right), \qquad R^*>0
\end{align}
Da $a,b,c>0$, kan denne ligning kun være sand, hvis $\alpha<c=0.005$. Antager vi dette, får vi, at modellen har netop én ligevægt $(K^*, R^*)$, hvor $K^*, R^*>0$, nemlig
\begin{align}
\begin{pmatrix}
K^* \\ R^* 
\end{pmatrix} = \begin{pmatrix}
\frac{1}{c} \\ \\ 
\frac{1}{b}\left( 1 - \frac{\alpha}{c} \right)
\end{pmatrix} = \begin{pmatrix}
200 \\ \\
20\left( 1 - 200\alpha \right)
\end{pmatrix}
\end{align}

Jeg udregner funktionalmatricen for $f$ som
\begin{align}
Df(R,K) = \begin{bmatrix}
r_k(1 + bR -2\alpha K) & & -r_KbK \\ \\
r_RcR & & r_R(cK - 1)
\end{bmatrix}
\end{align}
Jeg evaluerer funktionalmatricen i ligevægten $(K^*, R^*)$:
\begin{align}
Df(R^*,K^*) =  \\ \\
\begin{bmatrix}
r_k\left(1 + b\left(\frac{1}{b}\left( 1 - \frac{\alpha}{c} \right) \right) -2\alpha\frac{1}{c} \right) & & -r_Kb\frac{1}{c} \\ \\
r_Rc\left(\frac{1}{b}\left( 1 - \frac{\alpha}{c} \right) \right) & & r_R\left(c\frac{1}{c} - 1\right)
\end{bmatrix} = \\ \\
\begin{bmatrix}
\frac{-r_Ka}{c} & & \frac{-r_Kb}{c} \\ \\
\frac{r_R}{c}( c- a ) & & 0
\end{bmatrix}
\end{align}
Jeg bestemmer det karakteristiske polynomien for funktionalmatricen evalueret i ligevægten:
\begin{align}
|Df(R^*,K^*) - \lambda E| = \\ 
\left(\frac{-r_K\alpha}{c} - \lambda \right)(-\lambda) + \frac{r_Kr_R(c-\alpha)}{c} = \\
\lambda^2 + \frac{r_K\alpha}{c}\lambda + \frac{r_Kr_R(c-\alpha)}{c}
\end{align}
Da vi har, at $r_K, r_R, \alpha, c>0$ og yderligere, at $\alpha<c$, er det karakteristiske altså på formen
\begin{align}
\lambda^2 + A\lambda + B, \qquad A,B >0
\end{align}
Rødderne i et sådant polynomien har negative realdele. Altså har funktionalmatricen evalueret i ligevægten kun egenværdier med negative realdele. Altså er ligevægten stabil per sætning 9 på side 150 i kursusbogen Differentialligninger.

Altså har vi nu alt i alt vist, at hvis $\alpha<c$, så har vores model netop én ligevægt $(K^*, R^*)$, hvor $K^*, R^*>0$, nemlig
\begin{align}
\begin{pmatrix}
K^* \\ R^* 
\end{pmatrix} = \begin{pmatrix}
200 \\ \\
20\left( 1 - 200\alpha \right)
\end{pmatrix}
\end{align}
og denne ligevægt er stabil. 

Sætter vi $\alpha = 0.001$, får vi altså den stabile ligevægt
\begin{align}
\begin{pmatrix}
K^* \\ R^* 
\end{pmatrix} = \begin{pmatrix}
200 \\ \\
20\left( 1 - \frac{200}{1000} \right)
\end{pmatrix} = \begin{pmatrix}
200 \\ \\
16
\end{pmatrix}
\end{align}
hvilket svarer til det, som det numeriske plot også syntes at vise.

\subsection{Delopgave f}

Lad mig nu vise, at hvis et polynomien har formen
\begin{align}
p(\lambda) = \lambda^2 + A\lambda + B, \qquad A,B >0
\end{align}
så har alle dets rødder negative realdele. 

Lad $\lambda = a + b i$ være en rod til $p$. Altså gælder
\begin{align}
\lambda^2 + A\lambda + B = 0, \qquad A,B >0
\end{align}
hvilket er ækvivalent med
\begin{align}
\lambda^2 + A\lambda = -B, \qquad A,B >0
\end{align}
hvilket, da $B>0$, medfører
\begin{align}
\lambda^2 + A\lambda < 0, \qquad A >0
\end{align}
Antag nu, at $b=0$, hvilket med andre ord vil sige, at $\lambda = a + b i$ er et reelt tal. Da har vi, at 
\begin{align}
a^2 + Aa < 0, \qquad A >0
\end{align}
Da $A, a^2>0$, har vi altså, at $a<0$, hvilket vil sige, at $\lambda$ har en negativ realdel.

Antag nu i stedet, at $b\neq 0$. Da har vi, at 
\begin{align}
(a + bi)^2 + A(a + bi) < 0, \qquad A >0
\end{align}
hvilket per udregning er ækvivalent med
\begin{align}
(a^2 - b^2 + Aa) + (b(A + 2a))i\ <\ 0, \qquad A >0
\end{align}
Da $(a^2 - b^2 + Aa) + (b(A + 2a))i\ <\ 0$, følger det, at $(a^2 - b^2 + Aa) + (b(A + 2a))i \in \mathbb{R}$, da udsagnet $\gamma < 0$ slet ikke er defineret, hvis $\gamma$ ikke er et reelt tal. Altså følger det, at
\begin{align}
b(A + 2a) = 0, \qquad A >0
\end{align}
Da vi antaget, at $b\neq 0$, er dette ækvivalent med
\begin{align}
2a = -A, \qquad A >0
\end{align}
Da $A>0$, har vi altså nu, at $a<0$. Altså har $\lambda$ en negativ realdel.

Jeg har nu alt i alt vist det ønskede.

\section{Opgave 3: Ulve, får og græs}

Vi har modellen
\begin{align}
\begin{pmatrix}
G'(t) \\ 
F'(t) \\
U'(t)
\end{pmatrix} = f(G, F, U)
\end{align}
hvor 
\begin{align}
f(G, F, U) = \begin{pmatrix}
r_G G (1 - a G - bF) \\ 
r_F F (cG - 1 - dU) \\
r_U U (eF - 1)
\end{pmatrix}
\end{align}
hvor $a, e>0$, $b=0.001$, $c=0.01$, $d=0.2$, $r_G = 2$, $r_F=0.1$ og $r_U=0.7$.

\subsection{Delopgave a}

Det fremgår af leddene $eF>0$ og $-dU<0$, at ulve spiser får. Hermed har vi nemlig, at ulvenes væksthastighed bør vokse med antallet af får, hvilket afspejles i leddet $eF>0$ i udtrykket for $U'(t)$. Ligeledes har vi, at fårenes væksthastighed bør falde med antallet af ulve, hvilket afspejles i leddet $-dU<0$ i udtrykket for $F'(t)$. 

Det fremgår, at ulve ikke spiser græs ved, at antallet af ulve $U$ ikke indgår i udtrykket for græssets væksthastighed $G'(t)$, og tilsvarende at mængden af græs $G$ ikke indgår i udtrykket for ulvebestandens væksthastighed $U'(t)$. Hermed udtrykker modellen, at mængden af ulve og græs ikke  påvirker hinanden direkte, men kun indirekte gennem mængden af får. 

\end{document}